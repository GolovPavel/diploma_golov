% Примерный план:
% 1. Цель работы
% 2. Обзор литературы:
% 	- Найти книги где рассказывается про нуклиды в реакторе +
% 	- Найти информацию о переносе веществ в амосфере
% 	- Найти книги где пишется про метод конечных элементов
% 	- Найти книги, где можно прочитать про свойства метсности и коэффициенты
% 3. Обзор программных средств, используемых в работе.
% 	- Python
% 	- FEnics
% 	- GMSH
% 	- pygmsh, meshio
%   - UNK
\likechapter{Введение}

Производственная практика проходила в Национальном Исследовательском Ядерном Университете «МИФИ» на кафедре № 5. 

\textbf{Цель производственной практики.} Целью производственной практики является начало разработки модели \ac{ascro} 
(\acl{ascro}) для расчета концентрации осевших радиоактивных нуклидов на прилегающей к \ac{aes} (\acl{aes}) местности. 
В связи с поставленной целью в ходе работы решаются следующие задачи:

\begin{itemize}
	\item изучение радиоактивных нуклидов, которые образуются в результате ядерных реакций в топливе ядерного реактора 
		и теплоносителе первого контура;
	\item разработка модели образования радиоактивных нуклидов в топливе ядерного реактора с дальнейшим переходом в
		теплоноситель первого контура;
	\item разработка модуля анализа свойств местности, прилегающей к \ac{aes}, по данным топологических карт;
	\item разработка расчетной сетки, описывающей прилегающей к \ac{aes} местность;
	\item аппроксимация свойств местности на узлы расчетной сетки.
\end{itemize}

\textbf{Обзор литературы.} Проведем обзор имеющейся литературы в соответствии с поставленными задачами.

При изучении радиоактивных нуклидов, которые образуются в ядерном реакторе, важно учитывать не только продукты реакции 
деления $(n, f)$, но и радионуклиды, образующиеся в результате реакций превращения типа $(n, \gamma)$, $(n, 2n)$ на 
радиоактивных и стабильных изотопах в ядерном реакторе. <<Каждый нуклид, находящийся в работающем ядерном реакторе, в 
общем случае испытывает, кроме радиоактивного распада, ряд превращений, обусловленных взаимодействием с нейтронами: 
деление $(n, f)$, радиационный захват $(n, \gamma)$, реакцию $(n, 2n)$ и т.д.>> \citep[с.~6]{kolobashkin}.  
Авторы \cite{kolobashkin} выделяют боллее 650 радиоактивных нуклидов, массовые числа которых лежат в интервале от 72 до 
166, и около 60 актиноидов, которые генерируются в процессе работы ядерного реактора.

Радионуклидный состав и радиационные характеристики продуктов реакций деления и превращения в большинстве своем зависит 
от типа ядерного реактора и его особенностей. В источнике \cite{gusev_bio} описываются радионулиды, образующиеся в 
ядерных реакторах типа \ac{vver} (\acl{vver}) и \ac{rbmk} (\acl{rbmk}), а так же радиационные характеристики этих нуклидов. 
Основными радионуклидами, образующимися в топливе во время работы ядерного реактора и играющими существенную роль в 
создании радиационной обстановки вблизи \ac{aes}, являются \ac{irg} (\acl{irg}), а именно криптон (Kr) и ксенон (Xe), 
а так же изотопы иода (I).

Согласно \cite{gusev_bio, gusev_def, egorov}, помимо образования радиоактивных нуклидов в \ac{tvel}ах (\acl{tvel}), 
в ядерном реакторе присутствуют радионуклиды активационного и коррозионного происхождения. Как правило, такие нуклиды 
накапливаются в системе теплоносителя. Активность этой системы обусловлена следующими факторами:

\begin{itemize}
	\item собственная активность системы теплоносителя образуется из-за активации нейтронами ядер теплоносителя и 
		входящих в него естественных примесей;
	\item активность продуктов коррозии металлов (процесс коррозии в ядерном реакторе происходит на поверхности 
		конструктивных материалов \ac{az} (\acl{az}) и в системе теплоносителя);
	\item активность продуктов деления и актиноидов, проникающих через поврежденную или негерметичную оболочку \ac{tvel}ов.
\end{itemize}
Для реакторов, охлаждаемых легкой водой, основной вклад как по удельной активности, так и по мощности дозы $\gamma$ - 
излучения среди нуклидов активационного происхождения является радионуклид $^{16}\text{N}$, который образуется в
активной зоне в результате реакции $^{16}\text{O}$(n, p)$^{16}\text{N}$. Эта реакция является 
пороговой и проходит при энергии нейтронов выше 10 МэВ. Также не менее важными нуклидами активационного происхождения 
являются нуклид $^{17}\text{N}$, образуемый в реакции $^{17}\text{O}$(n, p)$^{17}\text{N}$ и нуклид $^{41}\text{Ar}$, 
образуемый в реакции $^{40}\text{Ar}$(n, $\gamma$)$^{41}\text{Ar}$.

При моделировании процесса переноса радиоактивных примесей в атмосфере необходимо учитывать множество факторов и 
метеорологических особенностей переноса веществ. <<Примесью в метеорологии обычно называют вещества, удаляемые в 
атмосферу в виде газов и  аэрозолей в процессе выброса>> \cite[с. 52]{gusev_bio}. После попадания в атмосферу такие 
вещества распространяются в результате ветрового переноса и турбулентной диффузии. 

Ветровой перенос примесей при их непрерывном истечении из источника создает своего рода струю выброса, в то время как 
при слабом или отсутствующем ветре вокруг источника выброса образуется облако примесей, обусловленное превалированием 
диффузии над ветровым переносом. Турбулентность диффузии в атмосфере обусловлена наличием в атмосфере беспорядочных 
завихрений различных размеров и форм, источником которых являются силы трения, возникающие при взаимодействии ветрового 
потока с землей, а так же потоки воздуха, распространяющиеся в вертикальном направлении вблизи нагретой поверхности. 
Взаимодействие атмосферных вихрей с облаком выброса зависит от отношения размеров вихрей и облака. \cite{gusev_bio}

При наличии ветра основной вклад в формировании турбулентности вносят силы трения ветровых потоков с поверхностью земли. 
Такой тип турбулентности называют \textit{механической}. Интенсивность данного типа турбулентности зависит от скорости 
ветра, но в большей степени от рельефа поверхности земли. Влияние рельефа поверхности на рассеяние примесей описывают 
величиной, называемой \textit{шероховатостью подстилающей поверхности} $z_0$. Согласно \cite{setton, bizova, berlyand}, 
высота подстилающей поверхности расчитывается по вертикальному профилю ветра в приземном слое воздуха при нейтральных 
условиях, однако на практике при метеорологических расчетах используют табличные значения \cite{mlyavaya}, приведенные 
в приложении А.

% TODO Почитать как оформляются приложения по госту, сделать приложение с табличкой с шероховатостями подстилающей 
% поверхнос



