\likechapter{Заключение}

Работа выполнена в соответствии с индвидуальным заданием по производственной практике, проходимой в Национальном 
Исследовательском Ядерном Университете «МИФИ» на кафедре № 5.

В рамках производственной практики происходила разработка модели \ac{ascro} для модуля тяжелых аварий ПМТ-3 энергоблока 
Калининской \ac{aes}. Получены новые и более углубленные знания об автоматизированных системах контроля радиационной 
обстановки вблизи \ac{aes} и приведено их краткое описание.

В ходе производственной практики были решены следующие задачи:
\begin{itemize}
	\item изучена литература, посвященная активации теплоносителя первого контура реакторной установки, а также 
	программные средства (UNK, FEniCS, Python, Gmsh, pygmsh), необходимые для разработки модели активации теплоносителя 
	первого контура и разработки модели \ac{ascro};
	\item представлены основные пути миграции радионуклидов на \ac{aes} при возникновении аварий и внештатных ситуаций. 
	Рассмотрены наиболее важные радионуклиды, которые образуются в процессе работы реактора и в случае внештатных 
	ситуаций могут попасть во внешнюю среду. Приведены цепочки образования наиболее важных радионуклидов в процессе 
	работы реактора. Разработана модель активации теплоносителя первого контура реакторной установки, позволяющая 
	произвести расчет концентраций наиболее значимых радиоактивных нуклидов, которые образуются в теплоносителе первого 
	контура реакторной установки активационным путем или мигрируют в теплоноситель из \ac{tvel}ов при разгерметизации 
	их оболочек;
	\item разработан программный модуль анализа свойств местности по данным топологических карт, позволяющий получить 
	информацию о типе прилегающей к \ac{aes} местности в каждой из её точек;
	\item разработана расчетная сетка, описывающая прилегающую к \ac{aes} местность, для дальнейшего решения уравнения 
	переноса радиоактивных примесей в атмосфере численным методом конечных элементов;
	\item разработан программный модуль, позволяющий аппроксимировать свойства прилегающей к \ac{aes} местности на узлы 
	расчетной сетки. 
\end{itemize}

В итоге можно сделать вывод о том, что в ходе производственной практики были достигнуты поставленные цели. В дальнейшем,
при выполнении дипломного проекта, планируется продолжить работу над проектом разработки модели \ac{ascro}.   