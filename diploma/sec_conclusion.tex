\likechapter{Заключение}

С каждым годом количество предприятий атомной промышленности стремительно растет. Увеличение количества предприятий 
сопровождается увеличением промышленных выбросов в атмосферу, среди которых присутствуют радиоактивные нуклиды. Повышение 
радиационной безопасности действующих \ac{aes}, контроль за уровнем радиационного фона на прилегающей к \ac{aes} местности
являются одними из главных задач в развитии атомной энергетики.

Для решения задачи контроля радиационной обстановки на прилегающей к \ac{aes} местности на действующих \ac{aes} 
используются системы \ac{ascro}. Основной целью \ac{ascro} является обеспечение руководства \ac{aes} информацией о 
радиационной обстановке вблизи \ac{aes}, которая способствует минимизации последствий для населения в случае превышения 
нормы радиационного фона или в случае радиационной аварии на \ac{aes}.

Для повышения квалификации персонала действующих \ac{aes} возникла потребность в разработке модели \ac{ascro}. В рамках 
дипломного проекта происходила разработка модели \ac{ascro} для полномасштабных тренажеров \ac{aes}. 

В ходе работы над проектом были решены следующие задачи:
\begin{itemize}
	\item изучена литература, посвященная активации теплоносителя первого контура реакторной установки, а также 
	программные средства (UNK, FEniCS, Python, Gmsh, pygmsh), необходимые для разработки модели активации теплоносителя 
	первого контура и разработки модели \ac{ascro};
	\item представлены основные пути миграции радионуклидов на \ac{aes} при возникновении аварий и внештатных ситуаций. 
	Рассмотрены наиболее важные радионуклиды, которые образуются в процессе работы реактора и в случае внештатных 
	ситуаций могут попасть во внешнюю среду. Приведены цепочки образования наиболее важных радионуклидов в процессе 
	работы реактора. Разработана модель активации теплоносителя первого контура реакторной установки, позволяющая 
	произвести расчет концентраций наиболее значимых радиоактивных нуклидов, которые образуются в теплоносителе первого 
	контура реакторной установки активационным путем или мигрируют в теплоноситель из \ac{tvel}ов при разгерметизации 
	их оболочек;
	\item разработан программный модуль анализа свойств местности по данным топологических карт, позволяющий получить 
	информацию о типе прилегающей к \ac{aes} местности в каждой из её точек;
	\item разработана расчетная сетка, описывающая прилегающую к \ac{aes} местность, для решения уравнения переноса 
	радиоактивных примесей в атмосфере численным методом конечных элементов;
	\item разработан программный модуль, позволяющий отображать свойства прилегающей к \ac{aes} местности на узлы 
	расчетной сетки;
	\item рассмотрено полуэмпирическое уравнение адвекции-диффузии, решение которого осуществляется методом конечных 
	элементов при моделировании переноса радионуклидов в атмосфере в разрабатываемой модели;
	\item разработан программный модуль, позволяющий расположить датчики гамма-излучения на прилегающей к \ac{aes} 
	местности в модели \ac{ascro} и рассчитать мощность эквивалентной дозы гамма-излучения в местах расположения 
	датчиков.
\end{itemize}

В итоге можно сделать вывод о том, что в ходе работы над выпускным квалификационным проектом, в соответствии с целью 
работы, были выполнены все поставленные задачи, связанные с разработкой модели \ac{ascro} для полномасштабных тренажеров.