\likechapter{Аннотация}

В работе происходила разработка модели автоматической системы контроля радиационной обстановки для полномасштабных 
тренажеров \ac{aes}. 

Ряд событий в прошлом, связанных с авариями на предприятиях атомной энергетики, показали необходимость создания систем 
\ac{ascro}. В современном мире системы \ac{ascro} являются неотъемлемой частью действующих \ac{aes}. С целью 
профессиональной подготовки оперативного персонала предприятий, возникла задача разработки модели 
\ac{ascro} для полномасштабных тренажеров действующих \ac{aes}.

В рамках выпускной квалификационной работы были решены следующие задачи:

\begin{itemize}
  \item разработка модели образования радиоактивных нуклидов в топливе ядерного реактора с дальнейшим переходом в 
  	теплоноситель первого контура;
  \item разработка модуля анализа свойств местности, прилегающей к \ac{aes}, по данным топологических карт;
  \item разработка расчетной сетки, описывающей прилегающую к \ac{aes} местность;
  \item отображение свойств местности на узлы расчетной сетки;
  \item разработка модуля расчета мощности эквивалентной дозы внешнего гамма - излучения.
\end{itemize}
