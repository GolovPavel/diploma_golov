Рассмотрим параметры уравнения адвекции-диффузии, описанного в разделе \ref{diffusion_model}, которые зависят от типа 
местности (подстилающей поверхности).

\textbf{Скорость ветра.} Зависимость скорости ветра от высоты имеет следующий вид \cite{wind_xy} (формула 
\ref{eq_wind_xy}):

\begin{equation}
	\label{eq_wind_xy}
    U(h) = U_0(\frac{h}{h_0})^k
\end{equation}

где:
\begin{description}
    \item $U_0$ --- скорость ветра на высоте флюгера;
    \item $h$ --- высота, на который определяется скорость ветра;
    \item $h_0$ --- высота флюгера (10 метров);
    \item $k$ --- степенной показатель, зависящий от типа подстилающей поверхности.
\end{description}

В работе \cite{wind_xy} приведена эмпирическая зависимость, позволяющую рассчитать значения степенного показателя $k$ в 
зависимости от класса шероховатости поверхности. Результаты расчета для используемых в модели \ac{ascro} типов 
подстилающих поверхностей представлены в таблице \ref{table_k}.

\begin{table}[ht!]
    \caption{Зависимость степенного показателя $k$ от степени шероховатости подстилающей поверхности \cite{wind_xy}.}
    \label{table_k}
	\setlength{\extrarowheight}{0.5mm}
	\centering
    \begin{tabular}{|M{0.2\textwidth}|M{0.5\textwidth}|M{0.18\textwidth}|}
    \hline Класс шероховатости & Характеристика ландшафта & Степенной показатель $k$ \\
    
    \hline 0 & Водная поверхность  & 0 \\
    
    \hline 1 & Открытые сельскохозяйственные земли с одиночными зданиями & 0,245 \\
    
    \hline 3 & Деревни, малые города, леса  & 0,37 \\
    
    \hline
    \end{tabular}
\end{table}

С помощью формулы \ref{eq_wind_xy} можно рассчитать продольную и поперечную составляющие скорости ветра. Вертикальную 
составляющую, так же называемую скоростью гравитационного осаждения, можно рассчитать по формуле \ref{eq_wind_z}.

