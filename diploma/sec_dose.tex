% План на главу
% 0) Фотонное излучение от радионуклидов
% 1) Описать про датчики в системе аскро и их характеристики (дальность, макс мощность и тд)
% 2) Описать, зачем в модели аскро нужны датчики
% 3) Описать формулу для расчета
% 4) Описать реализацию непосредственно в коде

\chapter{Разработка модуля расчета мощности дозы внешнего гамма-излучения}
\label{chapter_dose}

\section{Необходимость расчета мощности дозы гамма-излучения}

Радиоактивные нуклиды, которые могут попасть в атмосферу в резултате аврии на \ac{aes}, являются бета-активными. При 
протекании бета-распада результирующее ядро может оказаться в возбужденном состоянии. Возбуждение ядра снимается 
посредством испускания гамма-квантов.

Одной из главных задач современных систем \ac{ascro} на \ac{aes} является измерение значений мощности дозы фотонного 
излучения на местности. Значение мощности дозы фотонного излучения на местности необходимо знать для контроля 
радиацинной обстановки вокруг \ac{aes}, а также своевременной выдаче рекомендаций по принятию решений о защите 
населения.
