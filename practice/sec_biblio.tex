\begin{thebibliography}{99}
	\bibitem{kolobashkin} Радиационные характеристики облученного ядерного топлива: Справочник / В. М. Колобашкин, П. М. 
		Рубцов, П. А. Ружанский и др. М.: Энергоатомиздат, 1980.
	\bibitem{gusev_bio} Гусев Н. Г., Беляев В. А. Радиоактивные выбрсоы в биосфере: Справочник. --- 2-е изд., перераб. 
		и доп.---М.: Энергоатомиздат, 1991.---256 с.
	\bibitem{gusev_def} Защита от ионизирующих излучений. В 2-х т. Т.2. Защита от излучений ядерно-технических 
		установок: Учебник для вузов / Н. Г. Гусев, Е. Е. Коваленко, В. П. Машкович, А. П. Суворов; под ред. Н. Г. 
		Гусева --- 3-е изд., перераб. и доп. М.: Энергоатомиздат, 1990.
	\bibitem{egorov} Егоров Ю. А. Основы радиационной безопасности атомных электростанций: Учеб. пособие для вузов / 
		Под ред. Н. А. Доллежаля. М.: Энергоиздат, 1982.
	\bibitem{setton} Сеттон О. Г. Микрометеорология. Л.: Гидрометеоиздат, 1958.
	\bibitem{bizova_meteor} Метеорология и атомная энергия: Пер. с англ. / Под ред. Н. Л. Бызовой, К. П. Махонько. 
		Л.: Гидрометеоиздат, 1971.
	\bibitem{berlyand} Берлянд М. Е. Современные проблемы атмосферной диффузии и загрязнения атмосферы. Л.: 
		Гидрометеоиздат, 1975.
	\bibitem{mlyavaya} Млявая Г. В. Влияние параметров шероховатости подстилающей поверхности на скорость ветра // 
		Экология и география. Науки о жизни. 2014. № 2(323). С. 181-187.
	\bibitem{radio_transfer} Generic Models and Parameters for Assessing the Environmental Transfer of Radionuclides 
		from Routine Releases. Exposures of critical groups. IAEA Safety Series N 57. Vienna: IAEA. 1982.
	\bibitem{disper_atmos} Учет дисперсионных параметров атмосферы при выборе площадок для атомных электростанций. 
		Серия изданий по безопасности № 50-SG-53. Вена: МАГАТЭ, 1982.
	\bibitem{met_radio} Methodology for Evaluation the Radiological Consequences of Radioactive Effluents Released in 
		Normal Operations. Commission of the European Commities. Doc. N V/3865/79. 1979.
	\bibitem{general_exposure} General Principles of Calculation for the Radiation Exposure Resulting from Radioactive 
		Effluents in Exhaust Air and in Surface Water. Translation --- Safety Codes and Guides. Geselschaft fur reaktorsicherhait. Koln, 1980. N 11.
	\bibitem{laihtman} Лайхтман Д. Л. Физика пограничного слоя атмосферы.---2-е изд. Л.: Гидрометеоиздат, 1974. 
	\bibitem{bizova_scatter} Бызова Н. Л. Рассеяние примеси в пограничном слое атмосферы. М.: Гидрометеоиздат, 1974.
	\bibitem{disper_models} IAEA. Atmospheric Dispersion and Dose Calculation Techniques. Rep. UCRL-90765. Lawrence 
		Livermore Nat. Laboratory. University of California. Livermore, 1984.
\end{thebibliography}