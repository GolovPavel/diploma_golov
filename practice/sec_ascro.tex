\chapter{Автоматизированные системы контроля радиационной обстановки}

% Примерный план:
% 1) Существующие системы аскро
% 2) Состав аскро
% 3) Про АСРК что нить написать
% 4) Какие датчики есть
% 5) Заключение

\section{Необходимость \ac{ascro}}

С каждым годом количество промышленных предприятий в мире стремительно увеличивается. К сожалению, увеличение количества 
промышленных предприятий сопровождается увеличением промышленных выбросов в атмосферу, из-за чего происходит загрязнение 
окружающей среды. Развитие производств, связанных с атомной энергетикой, также связано с проблемой загрязнения 
окружающей среды, особенно в случае радиационных аварий. В связи с этим, одноиз из главных задач атомной энергетики 
является повышение радиационной безопасности действующих \ac{aes}, а также других \ac{oiae} 
(\acl{oiae}).

Даже при изолированной от окружающей среды технологии производства на предприятиях атомной энергетики, есть вероятность 
возникновения внештатных ситуаций, приводящих к радиоактивным выбросам и загрязнению окружающей среды. Примерами могут 
служить следующие радиационные аварии, произошедшие в прошлом:

\begin{itemize}
	\item взрыв ёмкостей с радиоактивными отходами на химкомбинате <<Маяк>> (Челябинская область, СССР) 29 сентября 
		1957 г.;
	\item авария на заводе <<Селлафильд>> (Уиндскейл, Великобритания) 10 октября 1957 г.;
	\item авария на \ac{aes} <<Tree Mile Island>> (штат Пенсильвания, США) 28 марта 1979 г.;
	\item авария на Чернобыльской \ac{aes} 26 апреля 1986 г.;
	\item авария на заводе Tokaimura (Токай, Япония) 30 сентября 1999 г.;
	\item авария на \ac{aes} Фукусима-1 (Окума, Япония) 11 марта 2011 г.
\end{itemize}

Примеры выше показывают важность и актуальность задачи радиационного контроля внешней среды вблизи \ac{aes}, 
осуществляемого автоматизированной системой контроля радиационной обстановки. 

\section{Цели и задачи \ac{ascro} \cite{elokhin}}

Основной целью \ac{ascro} является обеспечение руководства \ac{aes} информацией, которая способствует минимизации 
последствий радиационной аварии на \ac{aes}. 

Задачами системы являются:

\begin{itemize}
	\item оперативное обнаружение повышенного или аварийного выброса радиоактивных веществ;
	\item прогнозирование распространения радиоактивных выбросов, а также загрязнения окружающей среды;
	\item измерение значений мощности дозы фотонного излучения на прилегающей к \ac{aes} местности;
	\item оценка дозовых нагрузок на персонал и население;
	\item выдача рекомендаций по принятию решений о защите населения.
\end{itemize}

Функционирование системы должно осуществляться в режиме реального времени (on-line), что достигается за счет 
автоматизированного сбора данных при помощи установленных датчиков. Впоследствии, на основе полученных данных 
осуществляется прогностические расчеты с использованием математических моделей распространения радиоактивных примесей в 
атмосфере.

\section{Системы \ac{ascro} на различных этапах развития атомной энергетики \cite{elokhin}}

Развитие автоматизированных систем контроля радиационной обстановки происходило вместе с развитием средств контроля
(таких как детекторы ионизирующих излучений, анализаторы спектра $\alpha$, $\beta$, $\gamma$ излучений), вычислительных 
мощностей \ac{evm}, средств отображения, обработки, хранения и передачи информации, а также с развитием средств 
математического моделирования прогнозирования радиационной обстановки в атмосфере.

Автоматизированные системы радиационного контроля делят на три основных поколения.

К первому поколению относятся системы, разрабатываемые до 1960-х годов. Эти системы были построены, в основном, на 
основе автоматических пороговых детекторов, задачей которых являлось сигнализирование при превышении допустимого уровня 
загрязнения окружающей среды. Системы контроля первого поколения ориентировались в основном на осуществление контроля 
за загрязнением внешней среды, вызванным работой \ac{aes} в штатном режиме. Причиной развития такого подхода служил тот 
факт, что радиоактивные выбросы и сбросы \ac{aes} на столько малы за счет систем локализации и подавления активности, 
что они практически не изменяют радиационную обстановку внешней среды.

Системы второго поколения, развитие которых происходило в период с 1960-х до 1980-х годов, содержали в себе еще один 
важный компонент - вычислительный центр со специальным программным обеспечением, позволяющий на основе метеорологических 
данных атмосферы прогнозировать радиационные характеристики во внешней среде. 

Автоматизированные системы контроля радиационной обстановки третьего поколения развиваются по сей день в условиях 
интенсивного совершенствования мощностей вычислительной техники. Из-за развития мощностей становится возможным 
использование более усовершенствованных математических моделей переноса радиоактивных примесей в атмосфере, которые 
позволяют проводить более точное и детальное прогнозирование радиационных характеристик внешней среды.