% Необходим для установки нестандартного типа документа
\usepackage{extsizes}

% Настраивает отступы от границ страницы.
\usepackage{geometry}
\geometry{left=3cm}
\geometry{right=1cm}
\geometry{top=2cm}
\geometry{bottom=2cm}

% Делаем pdf файл search­able и copy-able
\usepackage{cmap} 

% Поддержка русского языка
\usepackage[T2A]{fontenc}
\usepackage[utf8]{inputenc}
\usepackage[russian]{babel}

% Устанавливаем кодировку исходных файлов
\usepackage[utf8]{inputenc}

% Устанавливаем стандартным шрифтом Times New Roman
\input{fonts/fonts_linux.tex}
\renewcommand{\rmdefault}{ftm}

% Для вставки картинок
\usepackage{graphicx} 

% Математические дополнения от АМС
\usepackage{amssymb,amsfonts,amsmath,amsthm} 

% Отделять первую строку раздела абзацным отступом тоже
\usepackage{indentfirst} 

% Улучшенное форматирование таблиц
\usepackage{makecell}
\usepackage{multirow} 

% Нижние подчеркивания
\usepackage{ulem} 
 
 % Полуторный межстрочный интервал
\usepackage{setspace}
\onehalfspacing

% Настройка заголовков документа
\input{preamble/headers}

% Настройка оглавления
% Настройка оглавления
\usepackage{tocloft}

% Оглавление стандартных секций
\usepackage{tocloft}
\renewcommand{\cfttoctitlefont}{\hspace{0.38\textwidth} \bfseries\MakeUppercase}
\renewcommand{\cftbeforetoctitleskip}{-1em}
\renewcommand{\cftchapfont}{\normalsize\bfseries \MakeUppercase{\chaptername} }
\renewcommand{\cftchapleader}{\cftdotfill{\cftdotsep}}
\renewcommand{\cftsecfont}{\hspace{31pt}}
\renewcommand{\cftsubsecfont}{\hspace{11pt}}
\renewcommand{\cftbeforechapskip}{1em}
\renewcommand{\cftparskip}{-1mm}
\renewcommand{\cftdotsep}{2}
\setcounter{tocdepth}{2} 

% Оглавление специальных секций (введение, заключение, ...)
\newcommand{\empline}{\mbox{}\newline}
\newcommand{\likechapterheading}[1]{ 
    \begin{center}
    \textbf{\MakeUppercase{#1}}
    \end{center}
    \empline}

\makeatletter
    \renewcommand{\@dotsep}{2}
    \newcommand{\l@likechapter}[2]{{\@dottedtocline{0}{0pt}{0pt}{\bfseries #1}{\bfseries #2}}}
\makeatother
\newcommand{\likechapter}[1]{    
    \likechapterheading{#1}    
    \addcontentsline{toc}{likechapter}{\MakeUppercase{#1}}}