\begin{center}
	\large
	\hfill \break \hfill \break \hfill \break \hfill \break
	ПОЯСНИТЕЛЬНАЯ ЗАПИСКА\\ ПО ВЫПУСКНОЙ КВАЛИФИКАЦИОННОЙ РАБОТЕ\\ НА ТЕМУ:\\[1.5cm]

	\normalsize
	РАЗРАБОТКА МОДЕЛИ АВТОМАТИЧЕСКОЙ СИСТЕМЫ \\ КОНТРОЛЯ РАДИАЦИОННОЙ ОБСТАНОВКИ ДЛЯ \\ПОЛНОМАСШТАБНЫХ ТРЕНАЖЕРОВ\\[6.0cm]
\end{center}

\noindent
\small
Студент \hspace{4.2cm} \underline{\hspace{4.2cm}} \hspace{0.7cm} Голов Павел Антонович\\

\noindent
Руководитель выпускной\\ квалификационной работы \hspace{0.85cm} \underline{\hspace{4.2cm}} \hspace{0.7cm} Щукин Николай Васильевич\\

\noindent
Консультант \hspace{3.35cm} \underline{\hspace{4.2cm}} \hspace{0.7cm} Гололобов Сергей Михайлович\\

\noindent
Рецензент \hspace{3.85cm} \underline{\hspace{4.2cm}} \hspace{0.7cm} \\

\noindent
И.о. зав. каф. \hspace{3.25cm} \underline{\hspace{4.2cm}} \hspace{0.7cm} Гераскин Николай Иванович\\[0.6cm]

\clearpage

\normalsize
\noindent
Специальность\\ (направление) \underline{14.05.01} \hspace{4.0cm} Группа \underline{С14-105}

\hspace{9.8cm}<<Утверждаю>>

\hspace{10cm} и.о.зав.каф.\\[0.1cm]

\hspace{7cm} \uline{\hspace{2.0cm}} \hspace{0.01cm} \uline{Гераскин Николай Иванович}

\hspace{7.5cm} \scriptsize (подпись) \hspace{2.0cm} (фамилия, имя, отчество)

\normalsize
\hspace{9cm} 01 ноября 2019 года\\[0.6cm]

\Large
\begin{center}
\textbf{ЗАДАНИЕ НА ДИПЛОМНЫЙ ПРОЕКТ\\
\normalsize
(выпускную квалификационную работу ВКР)} 
\end{center}

\normalsize
\noindent
1. Фамилия, имя, отчество студента

\noindent
\uline{Голов Павел Антонович}

\noindent
2. Тема проекта (ВКР)

\noindent
\uline{Разработка модели автоматической системы контроля радиационной обстановки для полномасштабных тренажеров}

\noindent
3. Срок сдачи студентом готового проекта

\noindent
\uline{до 24 января 2020 года}

\noindent
4. Место выполнения проекта

\noindent
\uline{НИЯУ МИФИ, кафедра № 5}

\noindent
5. Руководитель проекта \uline{Щукин Николай Васильевич, профессор, НИЯУ МИФИ}

\scriptsize \hspace{7cm} (фамилия, имя, отчество, должность, место работы)

\normalsize
\noindent
6. Консультант проекта \uline{Гололобов Сергей Михайлович, аспирант, НИЯУ МИФИ}

\scriptsize \hspace{7cm} (фамилия, имя, отчество, должность, место работы)

\clearpage

\normalsize
1. Цель работы: \ul{разработка модели автоматической системы контроля радиационной обстановки для полномасштабных 
тренажеров действующих АЭС.}

2. Задание:

\hspace{1cm} а) литература и обзор работ, связанных с проектом \ul{во введении необходимо провести обзор литературы, связанной с 
активацией теплоносителя первого контура, моделированием переноса радионуклидов в атмосфере, а также расчетом мощности 
эквивалентной дозы внешнего гамма-излучения. Сделать краткий обзор программных средств и вычислительных пакетов, предназначенных 
для моделирования переноса радионуклидов в атмосфере и решения дифференциального уравнения второго порядка в частных производных 
численным методом конечных элементов.}

\hspace{1cm} б) расчетно-конструкторская, теоретическая, технологическая часть \ul{ознакомиться с современными 
автоматическими системами контроля радиационной обстановки, их составом и принципом работы. Рассмотреть основные пути 
миграции радионуклидов на АЭС в случае аварийной ситуации, а также проанализировать наиболее важные радионуклиды,
образующиеся в процессе работы реакторной установки. Разработать модель активации теплоносителя первого контура 
реакторной установки. Рассмотреть уравнение адвекции-диффузии для моделирования переноса радионуклидов в атмосфере.}

\hspace{1cm} в) экспериментальная часть \ul{разработать расчетную сетку для решения полуэмпирического уравнения 
адвекции-диффузии методом конечных элементов. Произвести отображение свойств прилегающей к АЭС местности, от которых 
зависят параметры уравнения адвекции-диффузии, на узлы расчетной сетки. Произвести решение уравнения адвекции-диффузии 
методом конечных элементов. Создать модель выброса радиоактивного облака в атмосферу и рассчитать мощность эквивалентной 
дозы внешнего гамма-излучения на различном расстоянии от источника выброса.}

3. Отчетный материал проекта:

\hspace{1cm} а) пояснительная записка;

\hspace{1cm} б) графический материал (с указанием обязательных чертежей); \\
\ul{В главе 1 приведена структурная схема автоматических систем контроля радиационной обстановки. В главе 2 приведены 
схема миграции радионуклидов на АЭС в случае возникновения аварийной ситуации, цепочки образования наиболее важных 
радионуклидов с точки зрения радиационного воздействия на человека. В главе 3 приведена схема работы разработанного 
модуля анализа свойств местности по данным топологических карт, а также примеры маркировки расчетной сетки по типу 
подстилающей поверхности. В главе 4 приведены схема работы модуля генерации расчетной сетки, пример построения расчетной 
сетки при помощи разработанного модуля, а также примеры отображения свойств местности на узлы расчетной сетки. В главе 
5 приведена иллюстрация модели выброса радиоактивного облака для измерения мощности эквивалентной дозы внешнего гамма-излучения, 
а также график зависимости мощности эквивалентной дозы внешнего гамма-излучения от координаты расположения датчика 
гамма-излучения в рассматриваемой модели.}

\hspace{1cm} в) макетно-экспериментальная часть\\
\ul{В работе приведено описание и пример работы разработанного программного модуля анализа свойств местности, 
прилегающей к АЭС, модуля генерации расчетной сетки и отображения свойств местности на её узлы, а также модуля расчета 
мощности эквивалентной дозы внешнего гамма-излучения на прилегающей к АЭС местности. В приложении Д представлены листинги,
содержащие исходный код разработанных программных модулей.}

4. Консультанты по проекту (с указанием относящихся к ним разделов проекта)

\begin{table}[ht]
    \setlength{\extrarowheight}{1mm} 
    \centering
    \begin{tabular}{|M{0.21\textwidth}|M{0.21\textwidth}|M{0.21\textwidth}|M{0.21\textwidth}|}
    \hline
\multirow{2}{*}{Раздел} & \multirow{2}{*}{Консультант} & \multicolumn{2}{c|}{Подпись, дата} \\ \cline{3-4}
& & Подпись & Дата \\ \hline
Глава 5 & Гололобов С.М. & & \\ \hline
    \end{tabular}
\end{table}

\clearpage

\begin{center}
	\Large
	\textbf{Календарный план работы над проектом} \\
	\normalsize
	(составляется руководителем с участием студента в течение первой недели с начала дипломного проектирования)
\end{center}

\begin{table}[ht]
    \setlength{\extrarowheight}{1mm} 
    \centering
    \begin{tabular}{|M{0.05\textwidth}|M{0.24\textwidth}|M{0.16\textwidth}|M{0.20\textwidth}|M{0.20\textwidth}|}
    \hline
    № п/п & Наименование этапов работы & Сроки выполнения этапов & Степень готовности проекта в \% к объему работы
    	& Время выполнения \\ \hline
	1 & Изучение литературы & 01.11.2019-10.11.2019 & 20 & 9 дней \\ \hline
	2 & Разработка модуля расчета мощности эквивалентной дозы внешнего гамма-излучения (глава 5) & 10.11.2019-01.01.2020 & 70 & 52 дня \\ \hline
	3 & Оформление пояснительной записки, презентации & 01.01.2020-24.01.2020 & 90 & 23 дня \\ \hline
	4 & Подготовка речи & 24.01.2020-30.01.2020 & 100 & 6 дней \\ \hline
    \end{tabular}
\end{table}

\hfill \break
\noindent
Дата выдачи задания 1 ноября 2019 года

\hfill \break
\noindent
Руководитель дипломного проекта \underline{\hspace{2.5cm}} \uline{Щукин Николай Васильевич}

\scriptsize \hspace{7.5cm} (подпись)

\hfill \break
\noindent
\normalsize
Задание принял к исполнению \hspace{0.9cm} \underline{\hspace{2.5cm}} \uline{Голов Павел Антонович}

\scriptsize \hspace{7.5cm} (подпись)

\hfill \break
\noindent
\normalsize
1 ноября 2019 года

\normalsize
\clearpage