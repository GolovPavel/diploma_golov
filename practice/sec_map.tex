\chapter{Разработка модуля анализа свойств местности, прилегающей к \ac{aes}}

% План:
% 1) Зачем анализировать свойства местности (какие параметры зависят от типа местности, зачем они нам нужны и что означают)
% 2) Принцип анализа свойств местности
% 3) Разработка программного кода

\section{Модель переноса радиоактивных примесей в атмосфере} 
\label{diffusion_model}

Модель переноса радиоактивных примесей в атмосфере, которая используется при разработке проекта \ac{ascro}, основывается 
на решении полуэмперического уравнения адвекции-диффузии \cite{elokhin} методом конечных элементов. Рассмотрим это 
уравнение, описывающее перенос радиоактивной примеси в атмосфере (формула \ref{eq_diffusion}): 

\begin{equation}
    \label{eq_diffusion}
    \frac{\partial q}{\partial t} + div(\vec{U}q) + \sigma q = Dq + f
\end{equation}

с начальными условиями \ref{eq_diffusion_initials}:

\begin{equation}
	\label{eq_diffusion_initials}
    q(x, y, z, t)|_{t=0} = 0
\end{equation}

и граничными условиями \ref{eq_edges_1} --- \ref{eq_edges_3}:

\begin{equation}
	\label{eq_edges_1}
	q(x, y, z, t)|_{x=0}=0; \,\,\,\,\,\,  q(x, y, z, t)|_{y=0}=0;
\end{equation}
\begin{equation}
	\label{eq_edges_2}
	q(x, y, z, t)|_{x \rightarrow \infty}=0; \,\,\,\,\,\,  q(x, y, z, t)|_{y \rightarrow \infty}=0;
\end{equation}
\begin{equation}
	\label{eq_edges_3}
	q(x, y, z, t)|_{z \rightarrow \infty}=0; \,\,\,\,\,\, 
	k\frac{\partial q}{\partial z}|_{z=z_{0}} = (\beta - \omega)q|_{z=z_{0}}
\end{equation}

где:
\begin{description}
    \item $q$ --- концентрация радиоактивной примеси;
    \item $\vec{U}=u\vec{i} + v\vec{j} + w\vec{k}$ --- вектор скорости частиц воздуха; $\vec{i}, \vec{j}, \vec{k},$ 
    	- единичные векторы; $u, v, w$ - продольная, поперечная и вертикальная составляющие вектора скорости; 
    \item $D = \frac{\partial}{\partial x}(\mu \frac{\partial q}{\partial x}) 
    	+ \frac{\partial}{\partial y}(\mu \frac{\partial q}{\partial y})
    	+ \frac{\partial}{\partial z}(k \frac{\partial q}{\partial z})$ --- коэффициент диффузии; 
    \item $\sigma$ --- постоянная релаксации ($c^{-1}$);
    \item $\mu(x,y,z)$ --- продольно-поперечный коэффициент турбулентной диффузии;
    \item $k(x,y,z)$ --- вертикальный коэффициент турбулентной диффузии;
    \item $f$ --- источник радиоактивной примеси;
    \item $\beta$ --- скорость сухого осаждения;
    \item $\omega$ --- гравитационная скорость осаждения радиоактивной примеси;
    \item $z_{0}$ --- параметр шероховатости подстилающей поверхности.
\end{description}

\section{Необходимость анализа свойств местности}

Анализ свойств местности, прилегающей к \ac{aes}, является важной задачей при разработке модели \ac{ascro}. 

Как было показано в разделе \ref{diffusion_model}, уравнение адвекции-диффузии зависит от таких функций и параметров, 
как: скорость частиц воздуха, коэффициент диффузии, скорость сухого осаждения, скорость гравитационного осаждения и 
шероховатость подстилающей поверхности. Все перечисленные параметры явно или неявно зависят от типа местности (подробнее 
об этих параметрах описано в последующих разделах отчета), поэтому, перед построением расчетной сетки и аппроксимацией 
свойств местности на узлы расчетной сетки, необходимо разработать модуль, отвечающий за анализ свойств местности, 
прилегающей к \ac{aes}, по данным топологических карт.

\section{Принцип анализа свойств местности}

