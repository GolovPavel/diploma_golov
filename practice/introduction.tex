% Примерный план:
% 1. Цель работы
% 2. Обзор литературы:
% 	- Найти книги где рассказывается про нуклиды в реакторе 
% 	- Найти книги где пишется про метод конечных элементов
% 	- Найти книги, где можно прочитать про свойства метсности и коэффициенты.
% 3. Обзор программных средств, используемых в работе.
% 	- Python
% 	- FEnics
% 	- GMSH
% 	- pygmsh, meshio

\likechapter{Введение}

Производственная практика проходила в Национальном Исследовательском Ядерном Университете «МИФИ» на кафедре № 5. 

\textbf{Цель производственной практики.} Целью производственной практики является начало разработки модели АСКРО для
	расчета концентрации осевших радиоактивных нуклидов на прилегающей к АЭС местности.

В связи с поставленной целью в ходе работы решаются следующие задачи:

\begin{itemize}
	\item изучение радиоактивных нуклидов, которые образуются в результате ядерных реакций в топливе ядерного реактора 
		и теплоносителе первого контура;
	\item разработка модели образования радиоактивных нуклидов в топливе ядерного реактора с дальнейшим переходом в
		теплоноситель первого контура;
	\item разработка модуля анализа свойств местности, прилегающей к АЭС, по данным топологических карт;
	\item разработка расчетной сетки, описывающей прилегающей к АЭС местность;
	\item аппроксимация свойств местности на узлы расчетной сетки.
\end{itemize}

\textbf{Обзор литературы.} Проведем обзор имеющейся литературы в соответствии с поставленными задачами.