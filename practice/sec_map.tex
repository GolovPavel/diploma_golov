\chapter{Разработка модуля анализа свойств местности, прилегающей к \ac{aes}}

% План:
% 1) Зачем анализировать свойства местности (какие параметры зависят от типа местности, зачем они нам нужны и что означают)
% 2) Принцип анализа свойств местности
% 3) Разработка программного кода

\section{Модель переноса радиоактивных примесей в атмосфере}

Модель переноса радиоактивных примесей в атмосфере, которая используется при разработке проекта \ac{ascro}, основывается 
на решении полуэмперического уравнения адвекции-диффузии \cite{elokhin} методом конечных элементов. Рассмотрим это 
уравнение, описывающее перенос радиоактивной примеси в атмосфере (формула \ref{eq_diffusion}): 

\begin{equation}
    \label{eq_diffusion}
    \frac{\partial q}{\partial t} + div(\vec{U}q) + \sigma q = Dq + f
\end{equation}

с начальными условиями \ref{eq_diffusion_initials}:

\begin{equation}
	\label{eq_diffusion_initials}
    q(x, y, z, t)|_{t=0} = 0
\end{equation}

и граничными условиями \ref{eq_edges_1} --- \ref{eq_edges_3}:

\begin{equation}
	\label{eq_edges_1}
	q(x, y, z, t)|_{x=0}=0; \,\,\,\,\,\,  q(x, y, z, t)|_{y=0}=0;
\end{equation}
\begin{equation}
	\label{eq_edges_2}
	q(x, y, z, t)|_{x \rightarrow \infty}=0; \,\,\,\,\,\,  q(x, y, z, t)|_{y \rightarrow \infty}=0;
\end{equation}
\begin{equation}
	\label{eq_edges_3}
	q(x, y, z, t)|_{z \rightarrow \infty}=0; \,\,\,\,\,\, 
	k\frac{\partial q}{\partial z}|_{z=z_{0}} = (\beta - \omega)q|_{z=z_{0}}
\end{equation}

где ... TODO

\section{Необходимость анализа свойств местности}

Анализ свойств местности, прилегающей к \ac{aes}, является важной задачей при разработке модели \ac{ascro}.