% Необходим для установки нестандартного типа документа
\usepackage{extsizes}

% Настраивает отступы от границ страницы.
\usepackage{geometry}
\geometry{left=3cm}
\geometry{right=1cm}
\geometry{top=2cm}
\geometry{bottom=2cm}

% Делаем pdf файл search­able и copy-able
\usepackage{cmap} 

% Поддержка русского языка
\usepackage[T2A]{fontenc}
\usepackage[utf8]{inputenc}
\usepackage[russian]{babel}

% Устанавливаем кодировку исходных файлов
\usepackage[utf8]{inputenc}

% Устанавливаем стандартным шрифтом Times New Roman
\input{fonts/fonts_linux.tex}
\renewcommand{\rmdefault}{ftm}

% Для вставки картинок
\usepackage{graphicx} 

% Математические дополнения от АМС
\usepackage{amssymb,amsfonts,amsmath,amsthm} 

% Отделять первую строку раздела абзацным отступом тоже
\usepackage{indentfirst} 

% Улучшенное форматирование таблиц
\usepackage{makecell}
\usepackage{multirow} 

% Нижние подчеркивания
\usepackage{ulem} 
 
 % Полуторный межстрочный интервал
\usepackage{setspace}
\onehalfspacing

% Дополнительная работа с математикой
\usepackage{amsmath,amsfonts,amssymb,amsthm,mathtools} % AMS
\usepackage{icomma} % "Умная" запятая: $0,2$ --- число, $0, 2$ --- перечисление

% Настройка заголовков документа
\input{preamble/pre_headers}

% Настройка оглавления
\input{preamble/pre_toc}

% Настройка списков
% Оформление списков
\usepackage{enumitem}
\makeatletter
    \AddEnumerateCounter{\asbuk}{\@asbuk}{м)}
\makeatother
\setlist{nolistsep}
\renewcommand{\labelitemi}{-}
\renewcommand{\labelenumi}{\asbuk{enumi})}
\renewcommand{\labelenumii}{\arabic{enumii})}

% Настройка библиографии
\input{preamble/pre_biblio}

% Настройка списка сокращений
\usepackage{acro}
% Первое в спике сокращения идет короткое имя
\acsetup{first-style=short}
% Убираем стандартный заголовок у списка сокращений
\DeclareInstance{acro-title}{empty}{sectioning}{name-format =}
% Задаем между сокращениями и определениями
\DeclareAcroListStyle{mytabular}{table}{
  table=tabular,
  table-spec=@{}p{2cm}p{\dimexpr\textwidth-2cm-2\tabcolsep}@{},
}
\acsetup{list-style=mytabular}

% Сам список сокращений
\DeclareAcronym{az}{
  short=АЗ,
  long=активная зона
}

\DeclareAcronym{aes}{
  short=АЭС,
  long=атомная электростанция
}

\DeclareAcronym{ascro}{
  short=АСКРО,
  long=автоматическая система контроля радиационной обстановки
}

\DeclareAcronym{vver}{
  short=ВВЭР,
  long=водо-водяной энергетический реактор
}

\DeclareAcronym{irg}{
  short=ИРГ,
  long=инертные радиоактивные газы
}

\DeclareAcronym{pmt}{
  short=ПМТ,
  long=полномасштабный тренажер
}

\DeclareAcronym{rbmk}{
  short=РБМК,
  long=реактор большой мощности канальный
}

\DeclareAcronym{tvel}{
  short=ТВЭЛ,
  long=тепловыделяющий элемент
}

\DeclareAcronym{evm}{
  short=ЭВМ,
  long=электронно - вычислительная машина
}

% Настройка таблиц и рисунков
% Настройка таблиц и рисунков
\usepackage{multirow}
\usepackage[tableposition=top]{caption}

% Подписи к таблицам и рисункам
\DeclareCaptionLabelFormat{gostfigure}{Рисунок #2}
\DeclareCaptionLabelFormat{gosttable}{Таблица #2}
\DeclareCaptionLabelSeparator{gost}{~---~}
\captionsetup{labelsep=gost}
\captionsetup[figure]{labelformat=gostfigure}
\captionsetup[table]{labelformat=gosttable}
\captionsetup{justification=raggedright,singlelinecheck=false}

% Особый тип колонок
\newcolumntype{M}[1]{>{\centering\arraybackslash}m{#1}}

% Путь к картинкам
\graphicspath{{img/}}


% Пакеты с дополнительными символами
\usepackage{textcomp}
\usepackage{gensymb}

% Отступ в описании символов формул
\renewcommand*\descriptionlabel[1]{\hspace\leftmargin$#1$}