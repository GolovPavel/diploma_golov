\chapter{Модель активации теплоносителя первого контура}

\section{Миграция радионуклидов на \ac{aes}}
\label{sec_nuclides_migration}

Как и любое масштабное производство, \ac{aes} выбрасывает в атмосферу и окружающую среду вредные вещества, среди 
которых есть и радиоактивные. При нормальных условиях эксплуатации эти выбросы незначительны, так как современные 
атомные электростанции содержат множество систем очистки сбросов от радионуклидов, однако при нарушении работы 
какой-либо из систем \ac{aes} становится серьезным источником выбросов радионуклидов в атмосферу. 

Хотя принцип работы различных типов ядерных реакторов одинаков, их технологические схемы и устройства различны. В 
данном разделе рассмотрим образование радионуклидов с дальнейшим переходом в теплоноситель первого контура на примере 
реактора типа \ac{vver}.

Основные пути распространения радиоактивных нуклидов на \ac{aes} представлены на рисунке \ref{fig_nuclides_spread}.

\begin{figure}[ht]
\centering
	\includegraphics[width=16cm]{nuclides_spread}
	\captionsetup{justification=centering}
    \caption{Основные пути распространения радионуклидов на \ac{aes}.}
    \label{fig_nuclides_spread}
\end{figure}

Топливную таблетку в \ac{tvel}ах рассматривают как первый барьер распространения радиоактивных нуклидов в пределах 
активной зоны ядерного реактора. В результате реакции деления и захвата нейтронов в топливной таблетке накапливаются 
радионуклиды, изменяя состав, физико-химические и механические свойства топливной композиции. При температуре ниже 
1000 \degree C диоксид урана, который наиболее часто используется в качестве топлива в реакторах типа \ac{vver}, 
удерживает все радионуклиды, образующиеся в процессе работы реактора. При росте температуры ситуация существенно 
меняется, так как продукты захвата и деления становятся более подвижными \cite{leskin_vver}.

Между топливной таблеткой и оболочкой \ac{tvel}а присутствует небольшой зазор и газовая полость, предназначенные для 
накопления продуктов деления и активации, которым удалось покинуть пределы топливной таблетки.

Вторым барьером распространения радионуклидов является оболочка \ac{tvel}ов. В случае герметичной оболочки \ac{tvel}ов 
выход радионуклидов за пределы оболочки достаточно мал. В реальности, из за высоких тепловых и радиационных нагрузок и 
процессов коррозионно-усталостного типа оболочки теряют свою герметичность. Согласно \cite{kolpakov_tvel}, при 
эксплуатации ядерного реактора пределом безопасной эксплуатации по количеству и величине дефектов составляет 1 \% 
\ac{tvel}ов с дефектами типа газовой неплотности.

В случае разгерметизации топливной оболочки радионуклиды диффундируют через микротрещины в теплоноситель, находящийся 
в активной зоне реактора, который в дальнейшем переходит в первый контур реакторной установки. Более того, 
дополнительным источником радиоактивности в теплоносителе первого контура является его активация нейтронами. 

При эксплуатации \ac{aes} в нормальном режиме работы обеспечивается локализация радиоактивных продуктов деления, 
продуктов активации в реакторной установке и основных системах очистки от радиоактивных нуклидов. Парогенератор и 
трубопроводы первого и второго контуров не позволяют значительной части радионуклидов покинуть основные барьеры. Однако
при наличии микротрещин или протечек в парогенераторе радиоактивность первого контура может перейти в теплоноситель 
второго контура реакторной установки. В то же время, при протечках в трубопроводах первого или второго контура 
радионуклиды попадают в технические помещения, а далее загрязненный воздух через вентиляционные системы выбрасывается 
в атмосферу. Газообразные выбросы с \ac{aes} перед попаданием в атмосферу проходят сложную систему очистки, которая в 
свою очередь необходима для снижения активности выбросов, а далее попадают в окружающую среду через высокую трубу.

Помимо газообразных радиоактивных отходов при работе реактора выделяются жидкие и твердые отходы (рисунок 
\ref{fig_nuclides_spread2}). Твердыми радиоактивными отходами являются конструкционные материалы из активной зоны и 
первого контура, фильтры очистки установок, загрязненные инструменты, приборы и т.д. Твердые радиоактивные отходы после 
эксплуатации отправляются на захоронение. Жидкие радиоактивные отходы также образуются в результате эксплуатации 
\ac{aes}, в дальнейшем их очищают, разбавляют, фильтруют или концентрируют и хранят в специальных емкостях в жидком 
виде, однако при протечках в парогенераторе и конденсаторе радиоактивные отходы, мигрировавшие во второй контур 
реакторной установки, могут попасть в водоем-охладитель \cite{bekman_nuclear}. 

\begin{figure}[ht]
	\centering
	\includegraphics[width=10cm]{nuclides_spread2}
	\captionsetup{justification=centering}
    \caption{Схема образования газообразных, жидких и твердых отходов от \ac{aes} \cite{bekman_nuclear}.}
    \label{fig_nuclides_spread2}
\end{figure}

\section{Образование газообразных радионуклидов \ac{aes}}
\label{sec_gas_nuclides}

Рассмотрим наиболее важные радионуклиды, которые образуются в процессе работы реакторной установки и потенциально 
могут попасть в атмосферу путем газообразных выбросов \ac{aes}. 

Важную роль в формировании радиационной обстановке в районе выбросов \ac{aes} являются инертные радиоактивные газы. 
\ac{irg} попадают в теплоноситель при разгерметизации оболочек \ac{tvel}ов путем диффузии. Более десятка нуклидов 
инертных радиоактивных газов (криптона и ксенона) образуется в процессе деления нейтронами ядерного топлива 
\cite{bekman_nuclear}, а так же при распаде других продуктов реакции деления. Часть радионуклидов имеют либо малый 
период полураспадам (меньше минуты), либо вносят ничтожно малый вклад в суммарную активность, из-за чего их можно не 
учитывать в расчетах. Основные радионуклиды и их периоды полураспада \ac{irg}, образующиеся в процессе работы реактора, 
представлены в таблице \ref{table_irg}. В реакторах типа \ac{vver} инертные радиоактивные газы могут поступать в 
атмосферу путем утечки воды из первого контура реакторной установки или при утечке воды из второго контура реакторной 
установки при протечках в парогенераторе. 

\begin{table}[ht]
	\setlength{\extrarowheight}{1mm}
	\caption{Основные радионуклиды \ac{irg}, образующиеся в процессе работы реактора \cite{gusev_bio}.}
	\label{table_irg}
	\centering
    \begin{tabular}{|M{0.4\textwidth}|M{0.4\textwidth}|}
    \hline Нуклид & $\text{T}_{1/2}$ \\
    \hline $^{133}\text{Xe}$ & 5,21 суток \\
    \hline $^{135}\text{Xe}$ & 9,14 часов \\
    \hline $^{137}\text{Xe}$ & 3,9 минут \\
    \hline $^{138}\text{Xe}$ & 17,5 минут \\

    \hline $^{85}\text{Kr}$ & 10,76 лет \\
    \hline $^{87}\text{Kr}$ & 76 минут \\
    \hline $^{88}\text{Kr}$ & 2,8 часов \\   
    \hline 
    \end{tabular}
\end{table}

Стоит отметить, что радионуклиды, период полураспада которых намного меньшим кампания реактора ($^{137}\text{Xe}, 
^{87}\text{Kr}$) в ходе работы реактора быстро достигают состояния насыщения и их количество не меняется со временем. В 
то же время долгоживущие радионуклиды, период полураспада которых близок или превышает кампанию реактора 
($^{85}\text{Kr}$) накапливаются в топливе практически линейно со временем \cite{naumov_security}.

Не менее важными газообразными радионуклидами, попадающими в атмосферу, являются изотопы иода. Изотопы иода в ядерном 
реакторе образуются в результате реакции деления, а так же при распаде продуктов деления топлива. Пути попадания 
радиоактивных изотопов иода в атмосферу аналогичны инертным радиоактивным газам. Основные радионуклиды иода, 
образующиеся в процессе работы реактора, представлены в таблице \ref{table_iod}.

\begin{table}[ht]
	\setlength{\extrarowheight}{1mm}
	\caption{Основные радионуклиды иода, образующиеся в процессе работы реактора \cite{gusev_bio}.}
	\label{table_iod}
	\centering
    \begin{tabular}{|M{0.4\textwidth}|M{0.4\textwidth}|}
    \hline Нуклид & $\text{T}_{1/2}$ \\
    \hline $^{131}\text{I}$ & 8,05 суток \\
    \hline $^{132}\text{I}$ & 2,26 часов \\
    \hline $^{133}\text{I}$ & 20,3 часов \\
    \hline $^{134}\text{I}$ & 52 минут \\
    \hline $^{135}\text{I}$ & 6,68 часов \\   
    \hline 
    \end{tabular}
\end{table}

Из-за небольшого периода полураспада, в реакторе для вышеперечисленных радионуклидов достаточно быстро устанавливается 
равновесное состояние. Исключение составляет изотоп $^{129}\text{I}$ ($\text{T}_{1/2} = 1.57 \times  10^7$ лет), однако 
согласно \cite{bekman_nuclear} долгожвущий иод не обнаруживают в атмосфере и окружающей среде вокруг \ac{aes} и его 
выбросы во много раз меньше выбросов других радионуклидов иода.

Некоторые продукты деления ядер топлива, продукты распада \ac{irg} и радионуклиды с наведенной активностью образуют 
аэрозоли, которые попадают во внешнюю среду с воздушными потоками. Наиболее важные радионуклиды, которые входят в 
состав аэрозольных выбросов с реактора типа \ac{vver}, представлены в таблице \ref{table_aero}. 

\begin{table}[ht]
    \setlength{\extrarowheight}{1mm}
    \caption{Основные радионуклиды, входящие в состав аэрозолей, образующиеся в процессе работы реактора 
        \cite{bekman_nuclear}.}
    \label{table_aero}
    \centering
    \begin{tabular}{|M{0.4\textwidth}|M{0.4\textwidth}|}
    \hline Нуклид & $\text{T}_{1/2}$ \\
    \hline $^{88}\text{Rb}$ & 18 минут \\
    \hline $^{134}\text{Cs}$ & 2,07 лет \\
    \hline $^{137}\text{Cs}$ & 30,17 лет \\
    \hline $^{138}\text{Cs}$ & 33 минуты \\
    \hline $^{60}\text{Co}$ & 5,27 лет \\   
    \hline 
    \end{tabular}
\end{table}

В процессе работы реактора образуются активационные газы - изотопы $^{16}\text{N}$ и $^{41}\text{Ar}$. Изотоп 
$^{16}\text{N}$ образуется в реакторе из $^{16}\text{O}$ в ходе реакции (n, p) (в реакторе \ac{vver} кислород 
содержится либо в оксидном топливе, либо в теплоносителе), а $^{41}\text{Ar}$ образуется при облучения нейтронами 
изотопа $^{40}\text{Ar}$, растворенного в теплоносителе вместе с воздухом, в ходе реакции (n, $\gamma$). Периоды 
полураспада основных активационных газов представлены в таблице \ref{table_active_gas}.

\begin{table}[ht]
    \setlength{\extrarowheight}{1mm}
    \caption{Основные активационные газы, образующиеся в процессе работы реактора 
        \cite{bekman_nuclear}.}
    \label{table_active_gas}
    \centering
    \begin{tabular}{|M{0.4\textwidth}|M{0.4\textwidth}|}
    \hline Нуклид & $\text{T}_{1/2}$ \\
    \hline $^{41}\text{Ar}$ & 1,83 часов \\
    \hline $^{16}\text{N}$ & 7,13 секунд \\
    \hline 
    \end{tabular}
\end{table}

Радиоактивный углерод $^{14}\text{C}$ образуется в реакторе по трём основным каналам: активация нейтронами естественных 
карбидных солей в теплоносителе первого контура, содержащих $^{13}\text{C}$, в ходе (n, $\gamma$) реакции; активация 
азота $^{14}\text{N}$, находящегося в виде примеси в топливе, в ходе (n, p) реакции; активация кислорода, содержащегося 
в оксидном топливе и в теплоносителе первого контура, в ходе (n, $\alpha$) реакции. Период полураспада изотопа 
$^{14}\text{C}$ составляет 5730 лет \cite{bekman_nuclear}.

Газообразный тритий образуется так же по трём основным каналам: при тройном делении ядер топлива (маловероятный процесс, 
из-за чего в расчетах можно принебречь им); в результате активации нейтронами ядер $^{6}\text{Li}$ и $^{10}\text{B}$, 
которые растворены в теплоносителе первого контура, в ходе реакций $^{6}\text{Li}$(n, $\alpha$)$^{3}\text{H}$ и 
$^{10}\text{B}$(n, $2\alpha$)$^{3}\text{H}$; в ходе активации дейтерия, содержащегося в теплоносителе в качестве 
примеси. Период полураспада трития составляет 12,32 лет \cite{bekman_nuclear}.

\section{Активация теплоносителя радионуклидами, выходящими из под оболочки \ac{tvel}ов}

Как было сказано в разделе \ref{sec_gas_nuclides}, в случае разгерметизации оболочек \ac{tvel}ов в теплоноситель 
первого контура могут попасть следующие радионуклиды, образующиеся в результате реакции деления ядер топлива: изотопы 
иода ($^{131}\text{I}$, $^{132}\text{I}$, $^{133}\text{I}$, $^{134}\text{I}$, $^{135}\text{I}$), изотопы, входящие в 
состав аэрозолей ($^{134}\text{Cs}$, $^{137}\text{Cs}$, $^{138}\text{Cs}$, $^{88}\text{Rb}$, $^{60}\text{Co}$), инертные 
радиоактивные газы ($^{133}\text{Xe}$, $^{135}\text{Xe}$, $^{137}\text{Xe}$, $^{138}\text{Xe}$, $^{85}\text{Kr}$, 
$^{87}\text{Kr}$, $^{88}\text{Kr}$).

Изменение концентрации i-ого нуклида, который образуется под оболочкой твэла в результате деления, определяется формулой 
\ref{eq_fission_conc}.

\begin{equation}
    \label{eq_fission_conc}
    \frac{\partial c_{i}}{\partial t} = -\lambda_{i}c_{i} + \gamma_{i}\Sigma_{f}\phi - S_{i}
\end{equation}

где:
\begin{description}
    \item $\text{c}_i$ - концентрация i-го нуклида под оболочкой \ac{tvel}а;
    \item $\lambda_{i}$ -  постоянная распада i-ого нуклида;
    \item $\gamma_{i}$ - вероятность выхода i-ого нуклида при реакции деления;
    \item $\Sigma_{f}$ - макроскопическое сечение деления ядер топлива;
    \item $\phi$ - поток нейтронов;
    \item $\text{S}_{i}$ - вероятность выхода i-ого нуклида из-под оболочки \ac{tvel}а.
\end{description}

В левой части уравнения \ref{eq_fission_conc} - скорость изменения концентрации i-ого нуклида под оболочкой \ac{tvel}а, 
которая представлена в виде суммы трех слагаемых: скорости распада нуклида, скорости образования нуклида в результате 
деления ядер топлива и скорости увода нуклида через микротрещина разгерметизированных оболочек \ac{tvel}ов в 
теплоноситель первого контура.

Уравнение \ref{eq_coolant_balance} описывает баланс между концентрацией i-ого нуклида в теплоносителе первого контура 
без учета его очистки и выходом i-ого нуклида из-под оболочки \ac{tvel}ов:

\begin{equation}
    \label{eq_coolant_balance}
    \widetilde{S}_{i} = \lambda_{i}\widetilde{c}_{i}\frac{V_{coolant}}{V_{core}}
\end{equation}

где:
\begin{description}
    \item $\widetilde{c}_{i}$ - равновесная концентрация i-ого нуклида в теплоносителе первого контура;
    \item $\widetilde{S}_{i}$ -  равновесный выход i-ого нуклида из-под оболочки \ac{tvel}ов, при котором в 
        теплоносителе первого контура будет установлена концентрация $\widetilde{c}_{i}$ i-ого нуклида;
    \item $V_{coolant}$ - объем теплоносителя первого контура;
    \item $V_{core}$ - объем активной зоны.
\end{description}

Выход радионуклидов через оболочку зависит от множества параметров. Для упрощения ограничимся зависимостью выхода от 
концентрации примеси в \ac{tvel}ах, давления в теплоносителе первого контура и числа поврежден­ных \ac{tvel}ов. 
Зависимость выхода i-ого радионуклида из-под оболочки \ac{tvel}ов представлена в формуле \ref{eq_tvel_quite}:

 \begin{equation}
    \label{eq_tvel_quite}
    S_{i} = \widetilde{S}_{i}F_{i}(c_{i}, P, N_{failedRods})
\end{equation}

где:
\begin{description}
    \item $P$ - давление в теплоносителе первого контура;
    \item $N_{failedRods}$ - количество поврежденных \ac{tvel}ов;
    \item $F_{i}$ - функция, учитывающая отклонение значение выхода i-ого нуклида из-под оболочки \ac{tvel}а от 
        равновесного при изменении любого из её параметров.
\end{description}

Отметим, что функция $F_{i}(c_{i}, P, N_{failedRods})$ в модели определяется таким образом, что $\widetilde{S}_{i}$ 
соответствует равновесному выходу i-ого радионуклида при номинальном режиме работы реакторной установки. Иначе говоря, 
при отсутствии поврежденных \ac{tvel}ов, а так же при давлении в первом контуре, соответствующем номинальному давлению и 
достижении равновесного значения концентрации i-ого радионуклида под оболочкой \ac{tvel}а функция 
$F_{i}(c_{i}, P, N_{failedRods})$ равна 1.
