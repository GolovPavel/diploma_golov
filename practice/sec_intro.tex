% Примерный план:
% 1. Цель работы
% 2. Обзор литературы:
% 	- Найти книги где рассказывается про нуклиды в реакторе 
% 	- Найти книги где пишется про метод конечных элементов
% 	- Найти книги, где можно прочитать про свойства метсности и коэффициенты.
% 3. Обзор программных средств, используемых в работе.
% 	- Python
% 	- FEnics
% 	- GMSH
% 	- pygmsh, meshio
%   - UNK
\likechapter{Введение}

Производственная практика проходила в Национальном Исследовательском Ядерном Университете «МИФИ» на кафедре № 5. 

\textbf{Цель производственной практики.} Целью производственной практики является начало разработки модели АСКРО для
расчета концентрации осевших радиоактивных нуклидов на прилегающей к АЭС местности. В связи с поставленной целью в 
ходе работы решаются следующие задачи:

\begin{itemize}
	\item изучение радиоактивных нуклидов, которые образуются в результате ядерных реакций в топливе ядерного реактора 
		и теплоносителе первого контура;
	\item разработка модели образования радиоактивных нуклидов в топливе ядерного реактора с дальнейшим переходом в
		теплоноситель первого контура;
	\item разработка модуля анализа свойств местности, прилегающей к АЭС, по данным топологических карт;
	\item разработка расчетной сетки, описывающей прилегающей к АЭС местность;
	\item аппроксимация свойств местности на узлы расчетной сетки.
\end{itemize}

\textbf{Обзор литературы.} Проведем обзор имеющейся литературы в соответствии с поставленными задачами.

При изучении радиоактивных нуклидов, которые образуются в ядерном реакторе, важно учитывать не только продукты реакции 
деления $(n, f)$, но и радионуклиды, образующиеся в результате реакций превращения типа $(n, \gamma)$, $(n, 2n)$ на 
радиоактивных и стабильных изотопах в ядерном реакторе. <<Каждый нуклид, находящийся в работающем ядерном реакторе, в 
общем случае испытывает, кроме радиоактивного распада, ряд превращений, обусловленных взаимодействием с нейтронами: 
деление $(n, f)$, радиационный захват $(n, \gamma)$, реакцию $(n, 2n)$ и т.д.>> \citep[с.~6]{kolobashkin}.  
Авторы \cite{kolobashkin} выделяют боллее 650 радиоактивных нуклидов, массовые числа которых лежат в интервале от 72 до 
166, и около 60 актиноидов, которые генерируются в процессе работы ядерного реактора.

Радионуклидный состав и радиационные характеристики продуктов реакций деления и превращения в большинстве своем зависит 
от типа ядерного реактора и его особенностей. В источнике \cite{gusev} описываются радионулиды, образующиеся в ядерных 
реакторах типа \ac{vver} и \ac{rbmk}, а так же их радиационные характеристики.  Основными радионуклидами, образующимися 
в топливе во время работы ядерного реактора, являются криптон (Cr), ксенон (Xe) и иод (I).