% Необходим для установки нестандартного типа документа
\usepackage{extsizes}

% Настраивает отступы от границ страницы.
\usepackage{geometry}
\geometry{left=3cm}
\geometry{right=1cm}
\geometry{top=2cm}
\geometry{bottom=2cm}

% Делаем pdf файл search­able и copy-able
\usepackage{cmap} 

% Поддержка русского языка
\usepackage[T2A]{fontenc}
\usepackage[utf8]{inputenc}
\usepackage[russian]{babel}

% Устанавливаем кодировку исходных файлов
\usepackage[utf8]{inputenc}

% Устанавливаем стандартным шрифтом Times New Roman
% Зачем: Выбор внутренней TeX кодировки.
\usepackage[T2A]{fontenc}

% Зачем: Предоставляет свободный Times New Roman.
% Шрифт идёт вместе с пакетом scalable-cyrfonts-tex в Ubuntu/Debian

% пакет scalable-cyrfonts-tex может конфликтовать с texlive-fonts-extra в Ubuntu
% решение: Для себя я решил эту проблему так: пересобрал пакет scalable-cyrfonts-tex, изменив его имя. Решение топорное, но работает. Желающие могут скачать мой пакет здесь:
% https://yadi.sk/d/GW2PhDgEcJH7m
% Установка:
% dpkg -i scalable-cyrfonts-tex-shurph_4.16_all.deb

\usefont{T2A}{ftm}{m}{sl}

\renewcommand{\rmdefault}{ftm}

% Для вставки картинок
\usepackage{graphicx} 

% Математические дополнения от АМС
\usepackage{amssymb,amsfonts,amsmath,amsthm} 

% Отделять первую строку раздела абзацным отступом тоже
\usepackage{indentfirst} 

% Улучшенное форматирование таблиц
\usepackage{makecell}
\usepackage{multirow} 

% Нижние подчеркивания
\usepackage{ulem} 
 
 % Полуторный межстрочный интервал
\usepackage{setspace}
\onehalfspacing

% Настройка заголовков документа
% Настройка заголовков документа
\usepackage{titlesec}
\titleformat{\chapter}[display]
    {\filcenter}
    {\MakeUppercase{\chaptertitlename} \thechapter}
    {8pt}
    {\bfseries}{}
 
\titleformat{\section}
    {\normalsize\bfseries}
    {\thesection}
    {1em}{}
 
\titleformat{\subsection}
    {\normalsize\bfseries}
    {\thesubsection}
    {1em}{}
 
% Настройка вертикальных и горизонтальных отступов
\titlespacing*{\chapter}{0pt}{-30pt}{8pt}
\titlespacing*{\section}{\parindent}{*4}{*4}
\titlespacing*{\subsection}{\parindent}{*4}{*4}

% Настройка оглавления
% Настройка оглавления
\usepackage{tocloft}

% Оглавление стандартных секций
\usepackage{tocloft}
\renewcommand{\cfttoctitlefont}{\hspace{0.38\textwidth} \bfseries\MakeUppercase}
\renewcommand{\cftbeforetoctitleskip}{-1em}
\renewcommand{\cftchapfont}{\normalsize\bfseries \MakeUppercase{\chaptername} }
\renewcommand{\cftchapleader}{\cftdotfill{\cftdotsep}}
\renewcommand{\cftsecfont}{\hspace{31pt}}
\renewcommand{\cftsubsecfont}{\hspace{11pt}}
\renewcommand{\cftbeforechapskip}{1em}
\renewcommand{\cftparskip}{-1mm}
\renewcommand{\cftdotsep}{2}
\setcounter{tocdepth}{2} 

% Оглавление специальных секций (введение, заключение, ...)
\newcommand{\empline}{\mbox{}\newline}
\newcommand{\likechapterheading}[1]{ 
    \begin{center}
    \textbf{\MakeUppercase{#1}}
    \end{center}}

\makeatletter
    \renewcommand{\@dotsep}{2}
    \newcommand{\l@likechapter}[2]{{\@dottedtocline{0}{0pt}{0pt}{\bfseries #1}{\bfseries #2}}}
\makeatother
\newcommand{\likechapter}[1]{    
    \likechapterheading{#1}    
    \addcontentsline{toc}{likechapter}{\MakeUppercase{#1}}}

% Настройка списков
% Оформление списков
\usepackage{enumitem}
\makeatletter
    \AddEnumerateCounter{\asbuk}{\@asbuk}{м)}
\makeatother
\setlist{nolistsep}
\renewcommand{\labelitemi}{-}
\renewcommand{\labelenumi}{\asbuk{enumi})}
\renewcommand{\labelenumii}{\arabic{enumii})}

% Настройка библиографии
% Настройка библиографии
\usepackage{csquotes}
\usepackage[square,numbers,sort&compress]{natbib}
\renewcommand{\bibnumfmt}[1]{#1.\hfill} % нумерация источников в самом списке — через точку
\renewcommand{\bibsection}{\likechapter{Список литературы}} % заголовок специального раздела
\setlength{\bibsep}{0pt}

% Настройка списка сокращений
\usepackage{acro}
% Первое в спике сокращения идет короткое имя
\acsetup{first-style=short}
% Убираем стандартный заголовок у списка сокращений
\DeclareInstance{acro-title}{empty}{sectioning}{name-format =}
% Задаем расстояние между сокращениями и определениями
\DeclareAcroListStyle{mytabular}{table}{
  table=tabular,
  table-spec=@{}p{2cm}p{\dimexpr\textwidth-2cm-2\tabcolsep}@{},
}
\acsetup{list-style=mytabular}

% Сам список сокращений
\DeclareAcronym{az}{
  short=АЗ,
  long=активная зона
}

\DeclareAcronym{aes}{
  short=АЭС,
  long=атомная электростанция
}

\DeclareAcronym{ascro}{
  short=АСКРО,
  long=автоматизированная система контроля радиационной обстановки
}

\DeclareAcronym{vver}{
  short=ВВЭР,
  long=водо-водяной энергетический реактор
}

\DeclareAcronym{irg}{
  short=ИРГ,
  long=инертные радиоактивные газы
}

\DeclareAcronym{maed}{
  short=МАЭД,
  long=мощность амбиентного эквивалента дозы
}

\DeclareAcronym{oiae}{
  short=ОИАЭ,
  long=объекты использования атомной энергии
}

\DeclareAcronym{pmt}{
  short=ПМТ,
  long=полномасштабный тренажер
}

\DeclareAcronym{rbmk}{
  short=РБМК,
  long=реактор большой мощности канальный
}

\DeclareAcronym{tvel}{
  short=ТВЭЛ,
  long=тепловыделяющий элемент
}

\DeclareAcronym{evm}{
  short=ЭВМ,
  long=электронно - вычислительная машина
}