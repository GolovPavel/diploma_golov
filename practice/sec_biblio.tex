\begin{thebibliography}{9}
	\bibitem{kolobashkin} Радиационные характеристики облученного ядерного топлива: Справочник / В. М. Колобашкин, П. М. Рубцов, П. А. Ружанский и др. М.: Энергоатомиздат, 1980.
	\bibitem{gusev_bio} Гусев Н. Г., Беляев В. А. Радиоактивные выбрсоы в биосфере: Справочник. --- 2-е изд., перераб. и доп.
	---М.: Энергоатомиздат, 1991.---256 с.
	\bibitem{gusev_def} Защита от ионизирующих излучений. В 2-х т. Т.2. Защита от излучений ядерно-технических установок: Учебник для вузов / Н. Г. Гусев, Е. Е. Коваленко, В. П. Машкович, А. П. Суворов; под ред. Н. Г. Гусева --- 3-е изд., перераб. и доп. М.: Энергоатомиздат, 1990.
	\bibitem{egorov} Егоров Ю. А. Основы радиационной безопасности атомных электростанций: Учеб. пособие для вузов / Под ред. Н. А. Доллежаля. М.: Энергоиздат, 1982.
	\bibitem{setton} Сеттон О. Г. Микрометеорология. Л.: Гидрометеоиздат, 1958.
	\bibitem{bizova} Метеорология и атомная энергия: Пер. с англ. / Под ред. Н. Л. Бызовой, К. П. Махонько. Л.: Гидрометеоиздат, 1971.
	\bibitem{berlyand} Берлянд М. Е. Современные проблемы атмосферной диффузии и загрязнения атмосферы. Л.: Гидрометеоиздат, 1975.
	\bibitem{mlyavaya} Млявая Г. В. Влияние параметров шероховатости подстилающей поверхности на скорость ветра // Экология и география. Науки о жизни. 2014. № 2(323). С. 181-187.
\end{thebibliography}